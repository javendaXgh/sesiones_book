%\vspace{-2.0cm}
\pagenumbering{roman}
\thispagestyle{empty}
\begin{center}
	UNIVERSIDAD CENTRAL DE VENEZUELA\\
	FACULTAD DE CIENCIAS ECONÓMICAS Y SOCIALES\\
	ESCUELA DE ECONOMÍA\\

	\begin{figure}
						\centering
						  \includegraphics[height=.7\textwidth]{img/UCV.png}
  \end{figure}
  \vspace{1.5cm}
  \large{\textbf{TODO UN TRABAJÓN}}

  \vspace{3cm}
  Trabajo de Grado de Maestría presentado ante la \\
  ilustre Universidad Central de Venezuela por el\\
  Br. Perencejo de Tal para optar
  al título de \\Economista\\
  \vspace{0.5cm}
  Tutor: Dr. Karl Adán MarxSmith\\
  \vspace{1.5cm}
  Caracas - Venezuela\\
  Marzo 2025
\end{center}


\newpage
\thispagestyle{empty}
\large{\textbf{Resumen}}
La investigación  \emph{Determinantes de la Inversión Extranjera Directa (IED) y su Efecto en el Crecimiento Económico: Un Análisis de Panel para Países de América Latina (2000-2020)}

La persistencia de la pobreza intergeneracional es uno de los principales desafíos en Mesopotámia. Los programas de Transferencias Monetarias Condicionadas (TMC) han surgido como una herramienta de política clave para aliviar la pobreza a corto plazo e incentivar la inversión en capital humano. Esta investigación de pregrado tiene como objetivo principal estimar el impacto causal del programa "Crecer Mejor"  sobre las tasas de asistencia escolar y los controles de salud infantil en los hogares beneficiarios. Utilizando datos de la Encuesta Nacional de Hogares (ENH) y aplicando una metodología de Regresión Discontinua (RDD) basada en el puntaje de elegibilidad, se busca aislar el efecto del programa. Los resultados (hipotéticos) indican un incremento estadísticamente significativo de 8 puntos porcentuales en la asistencia escolar secundaria y un aumento del 15\% en la probabilidad de asistencia a controles médicos preventivos para los hogares justo por debajo del umbral de corte. Estos hallazgos sugieren que las TMC son efectivas para fomentar el capital humano, aunque su sostenibilidad a largo plazo requiere análisis adicionales.

\vspace*{2cm}

\textbf{Palabras Clave:} Crecimiento Económico, Inversión Extranjera Directa (IED), Datos de Panel, América Latina, Determinantes Macro.




\newpage
\thispagestyle{empty}
\large{\textbf{Abstract}}

The research \emph{Determinants of Foreign Direct Investment (FDI) and its Effect on Economic Growth: A Panel Data Analysis for Latin American Countries (2000-2020)}.

Attracting Foreign Direct Investment (FDI) is a central goal for emerging economies due to its potential to foster growth through technology transfer and job creation. However, the factors determining these flows and their actual impact remain debated. This thesis investigates the macroeconomic determinants of FDI and subsequently evaluates the effect of FDI on per capita GDP growth in 10 Latin American countries during the 2000-2020 period. Using a fixed-effects panel data model, the impact of variables such as trade openness, institutional stability, and inflation on FDI inflows is analyzed. The (hypothetical) results suggest that institutional stability and trade openness are the most robust determinants. Furthermore, a subsequent analysis using Instrumental Variables (IV) regression indicates that a 1 percentage point increase in FDI (as a percentage of GDP) is associated with an additional 0.3 percentage point growth in per capita GDP, ceteris paribus.


\vspace*{2cm}

\textbf{Keywords:} Economic Growth, Foreign Direct Investment (FDI), Panel Data, Latin America, Macro Determinants.

\thispagestyle{empty}


%\newpage


\setlength{\abovedisplayskip}{-5pt}
\setlength{\abovedisplayshortskip}{-5pt}
\thispagestyle{empty}

\newpage
\begin{center}
\large{\textbf{\emph{\Huge{Dedicatoria:}}}}
\end{center}
\thispagestyle{empty}
\vspace*{5cm}
\thispagestyle{empty}
\begin{center} \Large \emph{A fulana,  } \end{center}
\vspace*{1cm}
\begin{center} \Large \emph{perencejo.} \end{center}
\vspace*{1cm}
\begin{center} \Large {\emph{Todos los que andan por ahí.}} \end{center}



\newpage
\begin{center}
\large{\textbf{\emph{\Huge{Agradecimientos:}}}}
\end{center}
\thispagestyle{empty}
\vspace*{2cm}
\thispagestyle{empty}

- A mi mismo,

\newpage
\thispagestyle{empty}
\vspace*{5cm}
\hfill
\begin{minipage}{0.70\textwidth}
\begin{quote}
Los economistas pasan la mitad de su tiempo prediciendo que va a pasar y la otra mitad justificando porqué no pasó lo que dijeron que iba a pasar.\\
--- Naruto, \textit{El Libro de las Aventuras}, Ficciones
\end{quote}
\hspace*{2cm}

\begin{quote}
.
Un economista es alguien que ve algo funcionar en la práctica y se pregunta si funcionará en teoría.
--- Donald Knuth, \textit{The Art of Speak Pendejadas}, Volume 3
\end{quote}
\end{minipage}

\thispagestyle{empty}
\maketitle

\clearpage


